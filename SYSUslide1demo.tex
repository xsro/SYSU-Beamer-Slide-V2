\documentclass[]{SYSUslide1}
%------------------------------------------------------
%正式内容从此处开始
%------------------------------------------------------
\RequirePackage[]{overpic,pict2e,subfigure}
\title[中山大学]{基于hello world 程序的研究}
\subtitle{research on hello world program}
\author[皮艾迪]{皮艾迪、李倩茹、马鹏程}
\advisor{范巴可}
\institute[航空航天学院]{
  中山大学

  航空航天学院
}
\date[深圳 \today]{深圳 \today}


\begin{document}

%创建标题页
\frame{\titlepage}

%创建目录页
\begin{frame}
\frametitle{主要内容}
\tableofcontents
\end{frame}
%------------------------------------------------------------

\section{序言}

\begin{frame}[fragile]
    本ppt使用biblatex支持参考文献引用.
    可以使用\verb|\cite{origin}|来生成引用,如\cite{origin}。
    同时可以在尾注注明文献,使用方法如\verb|\footfullcite{origin}|,效果为\footfullcite{origin}。
    有时也会使用\verb|\footcite{origin}|来在脚注中生成参考文献的标号,如\footcite{origin}.

    文末使用\verb|\printbibliography|可以打印参考文献页。
    注意beamer中使用\verb|\verb|命令是不推荐的,使用时必须在frame中加上选项"fragile"。
\end{frame}

\section{研究对象与目的}


\section{编译过程}

\section{执行过程}

\begin{frame}[allowframebreaks]
    \printbibliography[]
\end{frame}

\section*{致谢}  
\begin{frame}
\vfill
\centering
\begin{beamercolorbox}[sep=8pt,center,shadow=true,rounded=true]{title}
  \Huge{恳请批评指正}
\end{beamercolorbox}
\vfill
\end{frame}
\end{document}